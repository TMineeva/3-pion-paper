\documentclass[aps,prl,reprint]{revtex4-1}
\usepackage{blindtext}
\usepackage{graphicx}  % needed for figures
\usepackage{dcolumn}   % needed for some tables
\usepackage{bm}        % for math
\usepackage{amssymb}   % for math


\begin{document}

\leftline{To be submitted to PRL}
\leftline{Comment to {\tt mineeva@jlab.org}}

\title{Unfolding hadronization mechanisms via pion multiplicity measurements at CLAS}
                             % (includes institutions and visitors)

\author{T.Mineeva$^{1}$}
\email{mineeva@jlab.org}
\author{W.K.Brooks$^{1}$, R.Dupre$^{2}$, H.Hakobyan$^{1}$ \it{ et.al}}
\affiliation{$^{1}$Universidad Tecnica Federico Santa Maria, Region de Valparaiso, Chile}
\affiliation{$^{2}$ Institut de Physique Nucl�aire, CNRS/IN2P3, Orsay, France}

\date{\today}

\begin{abstract}
In this paper we present  results of $\pi^{+}$,  $\pi^{-}$ and $\pi^{0}$ attenuation from a series of SIDIS measurements off D, C, Fe and Pb nuclei.
 The data were collected at Jefferson Lab using CLAS detector and a 5.014 GeV electron beam.
 The goal of this work is to provide an interpretation of measured multiplicity ratios and unveil the leading mechanisms of hadronization in cold nuclear medium, namely, 
 partonic energy losses versus hadron absorption.\end{abstract}

\maketitle

%\blindtext \cite{article-minimal}

Introduction.
Hadronization is the process that lays in the heart of QCD and refers to the formation of hadrons from quarks and gluons.  It is a common feature of high energy physics experiments that has been observed for over two decades and is intimately related to the phenomena of colour confinement. Global characterization of hadronization has been established in earlier experiments, however, solid understanding of the leading mechanisms remains to be uncovered. 

To gain experimental insight onto parton propagation and hadron formation mechanisms we employ tagged kinematics of deep-inelastic scattering and well-known properties of nuclei that allow for the study of hadronization process on the femtometer scale. By embedding this process in  a nuclei of increasing size, we can access the space-time development of hadronization. 

First measurements of well-identified hadrons were performed by HERMES at DESY ~\cite{bib:hermes} and stimulated a new wave of  theoretical efforts. HERMES data suggested different kinematical regimes wherein partonic mechanisms are likely to prevail over hadronic interaction, and vise versa. Still, the interplay between the two regimes remains dim.  CLAS data takes advantage of unique kinematical window in which time scales of the processes are of the order of atomic nuclei.  Offering two orders of magnitude more integrated luminosity, CLAS data enables extraction of multidimensional data allowing for a better resolution between model predictions.

The topics connected to propagation of color charge and formation of hadrons are important themes in the heavy-ion and Drell-Yan communities. In addition to their intrinsic interest for confinement-related properties of QCD, understanding  of hadronization in cold matter, once matured, can influence our understanding of, for example,  jet quenching  processes observed in high energy collisions. 

Theoretical background.
The dynamic mechanism leading to color neutralization of the struck parton can be described theoretically using, for example, pQCD technics. The formation of the hadrons and the corresponding scale of hadronization in cold nuclear matter, according to theoretical estimations, is that of the order of nuclear size which indicates that this process is non-perturbative. Since perturbative technics are not applicable, description of hadronization process relies on phenomenological models. For this reason, experimental data is crucial to elucidate between model predictions.

There is wide range of models that can claim relative success in describing the data, either in hot nuclear medium or cold medium or or both~\cite{bib:review}.  In this paper we will compare predictions for multiplicity ratios from the three models on the kinematic range of our data. The two models are based on distinctly opposite assumptions, namely partonic energy loss and hadron absorption, and one is based on the interplay of the above two. 

The idea of parton energy loss has been first discussed in the context of hot dense medium in the early eighties. The base assumption  of energy loss models is that the depletion of hadron yields on nuclear targets is due to parton loosing energy by medium-induced gluon radiation. Current framework, BDMPS formalism~\cite{bib:bdmps}, is based on pQCD calculations and is now developed for both hot and cold medium. Since nuclear density and geometry of cold QCD matter is well-known, unlike hot matter whose density rapidly drops when it expands, cold matter is an ideal testing ground for energy loss formalism.  Arleo model has been shown to give a good description of heavy-ion data such as RHIC and LHC~\cite{bib:arleo1}, and it broadly agrees with EMC and HERMES data~\cite{bib:arleo2}. The nature of disagreement indicates that hadronization effects play a role. Those effects can be contrasted by comparing predictions from energy loss models against pure hadron absorption models, in which case we refer to GiBUU model~\cite{bib:gibuu1}. The GiBUU model is based on the  string fragmentation and uses PYTHIA event generator to produce initial electron-nucleon interaction, and machinery of BUU transport model to predict the final state, specifically, by assigning the hadron cross section according to constituent quark model. An important feature of this work is that in order to describe HERMES data, GiBUU introduces concepts of hadron formation time.  This model describes fairy well one-dimensional multiplicities from HERMES on $\pi^{\pm}$, $\pi^{0}$ and$K^{\pm}$ at large $z>$0.4, however, its biggest challenge is with respect to simultaneous description of EMC data~\cite{bib:gibuu2}. The advantage of GiBUU model is that since it takes hadronic cross sections, it allows for the flavor separation of the observed multiplicities.

Finally, color dipole model is pQCD inspired model that considers medium-stimulated energy loss, which have small effect,  and emphasizes the role of hadronization.  In this framework, nuclear suppression mainly stems from quantum-mechanical interferences of hadron formation inside or outside the medium. In its original version it addresses leading hadrons ($z>$0.5) and describes fairly well existing HERMES data~\cite{bib:kop}. The model has extension to heavy ion collisions. 

Observables.
Hadronic multiplicity ratio is measured as the super ratio of the number of hadrons produced per deep-inelastic scattering event on the large nuclei compared to deuterium:
\begin{eqnarray}
\mathit{
R^h_\mathrm{A}\left({\nu},Q^2,z,p_{T})^{2}\right)=\frac
{\left.\frac{N_h({\nu},Q^2,z,p_{T})^{2})}{\left.N_e({\nu},Q^2)\right|_{\mathrm{DIS}}}\right|_{\mathrm{A}} }
{\left.\frac{N_h({\nu},Q^2,z,p_{T})^{2})}{\left.N_e({\nu},Q^2)\right|_{\mathrm{DIS}}}\right|_{\mathrm{D}} }
}
\end {eqnarray}

\noindent where $N_{h}$ is the yield of semi-inclusive hadrons in a given (${\nu},Q^2,z,p_{T}^{2}$) bin and $N_{e}$ is the yield of inclusive deep-inelastic scattering leptons in the same (${\nu},Q^2$) bin.

Experimental data.
The data was taken at Jefferson Lab using CEBAF Large Acceptance Spectrometer (CLAS) and 5.014 GeV electron beam.  The deuterium target, which provides normalization for the nuclear ratios, was exposed to the beam simultaneously with one of the solid targets (C, Fe, or Pb), thereby reducing the systematic uncertainties, such as, for example, efficiency of particle detection.

The kinematic region covers a range in $x_B$ greater than 0.1, which insures that valence quarks in the target nucleon are the main contributors in hadron production.

Kinematics range of the data is that of DIS with Q$^2$=1.0-4.1 GeV$^{2}$ and $\nu$=2.2-4.25 GeV; full range in $z$ and $p_T^{2}$ was available. See ref.\cite{bib:review} for a complete definition of variables. 
In our kinematics, $x_B$ greater than 0.1, which insures that valence quarks in the target nucleon are the main contributors in hadron production.
Analysis cuts include y =$\nu$/$\nu_{max} <$ 0.85 to decrease radiative effects and W$^2$ $>$ 4.0 GeV$^{2}$ to avoid resonance region.

Pion identification consisted of the standard CLAS instrumentation, which uses tracking and time-of-flight (TOF) systems, an electromagnetic calorimeter (EC) and a gas Cerenkov counter (CC). Identification of charged pions was done  using a combination of TOF and CC with the EC to reject positrons. Neutral pions were measured in their two-photon decay mode in the EC with background subtraction. 
Data is fully corrected for detector specifics (vertex correction, energy corrections), acceptance effects and radiative corrections on nuclear targets. 
Error bars include statistical errors and systematical. 
 
In this paper we will present the dependence of multiplicities on A$^{1/3}$. HERMES data showed a non-linear effect of this behavior, which has implications for model predictions. In scenario of hadron absorption models, the probability of prehadron absorption is expected to increase with time and distance as it transverses nuclear medium, and that would be indicated by A$^{1/3}$ dependence. 
On the other hand, BDMPS-based energy loss models predict that the rate at which parton losses energy is proportional to L$^{2}$ and therefore multiplicities should exhibit A$^{2/3}$, this, however, may not hold at low energies.  

We will present one-dimensional multiplicities for all three pions as a function of one of the four variables  (${\nu}$, Q$^2$, $z$ or $p_{T}^{2}$) and two-dimensional multiplicities as a function of 
($z$, $p_{T}^{2}$). Although CLAS coverage in $\nu$ is more limited then HERMES, it offers larger statistic and allows for quantitative comparison with HERMES data.  
The dependence of ($z$, $p_{T}^{2}$) is illustrative of Cronin effect, for which a significant increase is observed for high $p_{T}^{2}$ hadrons with a strong dependence on $z$.  
 

Conclusion.
The general features of the multiplicity ratio established by HERMES and well reproduced by CLAS data are suppression of $R^h_\mathrm{A}$ at higher values of $z$ that systematically increases for larger nuclei and an enhancement of $R^h_\mathrm{A}$ at high $p_T^{2}$  in analogy with the Cronin effect. Dependence on Q$^{2}$ and $\nu$, although visible, is small in our data. 

It is expected from most models, that the results for all three pions,  $\pi^{\pm}$ and $\pi^{0}$, are very similar. By contrasting to data two model predictions based on parton energy loss (Arleo model) and 
hadron absortion model (GiBuu), we will be able to pinpoint leading mechanisms providing the model predictions are different. If model predictions are not distinctly different, we will need
higher dimensionality which will be address by CLAS12 approved experiment. 

Acknowledgement. Grants numbers. Acknowledgement to people (D.Gaskell).

\bibliographystyle{apsrev4-1} % Tell bibtex which bibliography style to use
\bibliography{xampl} % Tell bibtex which .bib file to use (this one is some example file in TexLive's file tree)
\begin{thebibliography}{9}
\bibitem{bib:hermes}
{\bf HERMES} Collaboration A. Airepetian, {\it et al.}, "Hadronization in semi-inclusive deep-inelastic scattering on nuclei" Nucl. Phys. {\bf B780} (2007). \\
{\bf HERMES} Collaboration A. Airepetian, {\it et al.}, "Multidimensional study of hadronization in nuclei", arXiv:0811.1674v2 (2011)
\bibitem{bib:review} 
A. Accardi, F. Arleo, W. K. Brooks, D. D$'$Enterria and V. Muccifora. "Parton Propagation and Fragmentation in QCD Matter", Riv. Nuovo Cim., vol. 32 (2010)
\bibitem{bib:arleo1}
 F.Arleo "Quenching of hadron spectra at Heavy Ion Colliosions at the LHC", arXiv:1703.10852 (2017)\\
 F.Arleo "Tomography of cold and hot QCD matter: tools and diagnosis" 	arXiv:hep-ph/0210104 (2002)
\bibitem{bib:arleo2}
\ F.Arleo "Quenching of hadron spectra in DIS on nuclear targets", {\it Eur.Phys. J.}, C30 (2003)
\bibitem{bib:gibuu1}
 T. Falter, W. Cassing, K. Gallmeister, U. Mosel, "Hadron formation and attenuation in deep inelastic lepton scattering off nuclei" , arXiv:nucl-th/0303011 (2004) 
\bibitem{bib:gibuu2}
K. Gallmeister and U. Mosel,"Time dependent hadronization via HERMES and EMC data consistency" Nucl. Phys. A801 (2008) 
\bibitem{bib:kop}
 B. Z. Kopeliovich, J. Nemchik, E. Predazzi and A. Hayashigaki "Nuclear Hadronization: Within or Without?", Nucl. Phys. A740 (2004)
\bibitem{bib:bdmps}
R.~Baier, Y.L.~Dokshitzer, A.H.~Muller, D.~Schiff, "Radiative energy loss of high energy quarks and gluons in a finite volume quark-gluon plasma"  Nucl. Phys.{\bf B531} (1997)
\end{thebibliography}

\end{document}